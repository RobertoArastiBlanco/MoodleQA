\capitulo{7}{Conclusiones y Líneas de trabajo futuras}

\section{Líneas de trabajo futuras}
\begin{itemize}
	\item
	El método que utiliza la aplicación para realizar la comprobación de si el profesor responde a las dudas de los alumnos es una solución preliminar e incompleta. Se tiene como objetivo a futuro encontrar una forma más fiable de determinar qué es una duda y cuándo ha sido resuelta.
	El uso de modelos basados en el procesamiento del lenguaje natural puede ser un campo exploratorio que permita poder clasificar un mensaje del foro como una respuesta de dudas de un profesor. Pensamos que el diseño experimental y el posterior análisis de un clasificador con este cometido es suficientemente complejo para ser considerado un TFG por sí mismo.
	\item
	De las cinco consultas definidas referentes a los cuestionarios Moodle, no hay ninguna implementada en la aplicación, y esto se debe a que al menos cuatro de ellas no están soportadas por la API de servicios web de Moodle.
	Una posible solución a estas comprobaciones es el uso de técnicas de web scraping, pero teniendo en cuenta que las páginas que contengan la información deseada pueden variar en función del servidor que aloje la plataforma, no es una solución trivial.
	\item
	Por el momento la aplicación solo integra cursos Moodle, pero sería conveniente que la aplicación permitiera analizar cursos online de otros LMS como Blackboard o Edmodo. Sin embargo, realizar los cambios para esto supondría adaptarse a las APIs de servicios correspondientes suponiendo que sean lo suficientemente parecidas, y en caso de no poder acceder a la información necesaria mediante servicios web, habría que implementar otras formas de acceder a la información necesaria.
	\item
	Los plugin de Moodle con los que comparo la aplicación tienen más idiomas disponibles, esto se debe a que están dispuestos de forma que cualquiera pueda aportar sus propias traducciones, sin embargo, el internacionalizar la aplicación la haría más competitiva.
	\item
	La lista de cursos que muestra la aplicación es la lista de los cursos en los que se encuentra matriculado el usuario con independencia de su rol o los permisos que tenga. Sería interesante seguir mostrando los mismos cursos pero sin resaltar en caso de que no se tengan los permisos necesarios para realizar las consultas para poder contactar al administrador en caso de problemas con los permisos.
\end{itemize}
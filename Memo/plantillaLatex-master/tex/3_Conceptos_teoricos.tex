\capitulo{3}{Conceptos teóricos}


\section{Definiciones básicas}

\begin{itemize}
	\item \textbf{E-learning:}
	El e-learning es la enseñanza impartida por medios electrónicos como internet, plataformas virtuales, medios audiovisuales…etc.
	\item \textbf{Moodle:}
	Moodle es una plataforma de aprendizaje que permite a los profesores crear entornos de aprendizaje altamente personalizables. Fue creado por Martin Dougiamas que publicó su primera versión el 20 de agosto de 2002\cite{dougiamas2002interpretive}.
	\item \textbf{Calidad:}
	La calidad se puede definir como la capacidad de satisfacer una serie de necesidades y en el caso del e-learning se trata de las necesidades educativas del alumno, como la calidad del material educativo o la ayuda a la comprensión de este.
\end{itemize}

\section{Buenas prácticas}
\subsection{Retroalimentación a tiempo}
Es importante que el docente responda a las dudas a tiempo, y también que el alumno conozca sus resultados en un tiempo razonable, y esto es por dos razones: la primera, que el alumno no pierda la motivación si tiene buenos resultados, y la segunda, que se esfuerze más si tiene resultados mejorables.
\subsection{Opinión de los alumnos}
La evaluación es una parte fundamental del ciclo del diseño instruccional.
Las encuestas permiten encontrar los puntos fuertes (para no alterarlos) y los débiles (para solventarlos).

\subsection{Cuestionarios}
Es muy útil tener en cuenta las estadísticas de los cuestionarios, aunque para algunas de ellas, como suele ocurrir con la estadística, son necesarios suficientes intentos por parte de los estudiantes para tener una validez razonable. Una de esas estadísticas es el índice de facilidad, que indica el porcentaje medio de la puntuación que obtienen los alumnos de una pregunta, indicando su dificultad. Otra estadística interesante es la calificación aleatoria estimada, que expresa el resultado medio de responder a boleo una pregunta de tipo test; en un caso ideal este valor sería de 0\%, si no, la puntuación de esa pregunta sería injusta. Por último, la eficiencia discriminativa indica como de buena es una pregunta discriminando entre alumnos que saquen buena nota en el cuestionario y aquellos que saquen peor nota. Tener en cuenta estos valores puede ayudar a conseguir cuestionarios más justos y efectivos.

También es importante que las preguntas tengan una retroalimentación asignada para que los alumnos aprendan del resultado en la revisión.

Es recomendable ajustar el tiempo que se permite para la realización de los cuestionarios a partir del tiempo que tomó realizarlo en anteriores ocasiones para dificultar el fraude y tampoco quedarse corto y no dejar a los alumnos terminar.

\subsection{Trabajo en equipo}
Obtener las competencias necesarias para el trabajo en equipo es fundamental para el alumno, sobretodo en la universidad, que pretende preparar a sus estudiantes para un entorno laboral. Es muy recomendable que un curso contenga actividades con entrega por grupos.

\subsection{Claridad y corrección}
La corrección y la claridad de la información en un curso ayuda a evitar que el alumno se pierda en las formas y entienda mejor lo que se pide de él.
El calificador debería mostrar de forma concisa el peso de cada procedimiento, no tener demasiadas categorías dentro de otras y si las notas máximas de cada actividad coinciden, leer los resultados se vuelve mucho más sencillo.

\subsection{Apuntes actualizados}
Aparte de que viene bien revisar los materiales proporcionados a los alumnos en busca de errores y para añadir novedades, el no hacerlo en mucho tiempo da al alumno una imagen de pasotismo por parte del profesor, y en mi experiencia como alumno he llegado a vivir esa situación.


\section{Marco de referencia de calidad de MOOQ}
Ahora paso a explicar las tres dimensiones del marco de referencia de calidad del e-learning de MOOQ \cite{stracke2018quality}. MOOQ es la allianza europea por la calidad de los MOOC (Massive Open Online Courses). Un MOOC varía respecto al caso de e-learning para el que pretendemos crear la aplicación, pero al reducir la calidad del diseño instruccional a sus fases, roles y perspectivas aplicado a un tipo de e-learning es lo suficientemente válido como para sernos útil.
\subsection{Fases}
\begin{itemize}
	\item \textbf{Análisis:}
	En esta fase se definen los objetivos, el contexto, y los recursos (docentes, tiempo, presupuesto...) para la ejecución para comprender la situación inicial.
	\item \textbf{Diseño:}
	En esta fase se define lo que se planea hacer a partir de los resultados de la fase de análisis, como por ejemplo el enfoque de enseñanza que se piensa llevar a cabo.
	\item \textbf{Implementación:}
	En esta fase se define de qué manera se van a llevar a cabo los planes descritos en la fase de diseño, como por ejemplo de qué manera se va a producir el contenido.
	\item \textbf{Realización:}
	Esta fase es en la que se interactúa con el alumno, se gestionan los problemas técnicos y las dudas de los alumnos, además de evaluar su aprendizaje.
	\item \textbf{Evaluación:}
	En esta última fase se evalúa la calidad del resto de fases mediante encuestas, entrevistas, u otros medios.
\end{itemize}

\imagen{CicloFases.png}{Ciclo de fases según el marco de MOOQ}

\subsection{Roles}
Los roles son conjuntos de responsabilidades asumidos por una o más personas. Una persona podría tomar dos o más roles dado el caso (diseñador y facilitador, por ejemplo).
\begin{itemize}
	\item \textbf{Diseñadores:}
	Los encargados de decidir de qué forma se van a impartir el curso y generan el contenido (autores, expertos en el tema, diseñadores instruccionales).
	\item \textbf{Facilitadores:}
	Son aquellos que conocen la materia a enseñar y son capaces de explicarlo y dar feedback, además de seguir el aprendizaje de los alumnos.
	\item \textbf{Proveedores:}
	Son los encargados de proveer los medios digitales para llevar a cabo el e-learning (programadores, diseñadores y desarrolladores de software).
\end{itemize}


\subsection{Perspectivas}
\begin{itemize}
	\item \textbf{Pedagógica:}
	El punto de vista que se centra en el contenido y en el aprendizaje por parte del alumno. Los procesos relacionados tienen que ver con el contenido, el feedback, y la evaluación de los alumnos.
	\item \textbf{Tecnológica:}
	El punto de vista que se centra en las necesidades tecnológicas del curso. La mayoría de los procesos está relacionada con esta perspectiva debido a la naturaleza del e-learning.
	\item \textbf{Estratégica:}
	El punto de vista que se centra en la consecución de los objetivos del curso dentro del tiempo y presupuesto establecidos para este. Los procesos relacionados tienen que ver con los objetivos, conceptos y el contexto en el que se enseña (presupuesto, demanda, competencia...).
\end{itemize}

\section{Consultas}
En este apartado definiré qué información intentamos conseguir de cada fase (nos vamos a centrar en las fases de diseño, implementación y realización), rol y perspectiva mediante distintos métodos. La mayoría de consultas provienen de la lista de comprobación de UBUCEV de asignaturas virtuales.
Leyenda:
Responsabilidad: R=Responsable,X=Involucrado
Perspectivas: P=Pedagógica, T=Tecnológica, E=Estratégica

\subsection{Diseño:}
\begin{table}[H]
\resizebox{\textwidth}{!}{%
	\begin{tabular}{|l|l|l|l|l|l|}
		\hline
		Consulta                                                                                                                             & Perspectivas & Diseñador & Facilitador & Proveedor & Proceso \\ \hline
		\begin{tabular}[c]{@{}l@{}}Las opciones\\ de progreso\\ del estudiante\\ están activadas\end{tabular}                                & P            & R         & X           &           & D-5     \\ \hline
		\begin{tabular}[c]{@{}l@{}}Se proporcionan\\ contenidos en\\ diferentes formatos\end{tabular}                                        & PT           & R         & X           & X         & D-4     \\ \hline
		\begin{tabular}[c]{@{}l@{}}El curso tiene\\ grupos\end{tabular}                                                                      & P            & R         & X           & X         & D-3     \\ \hline
		\begin{tabular}[c]{@{}l@{}}El curso tiene\\ actividades\\ grupales\end{tabular}                                                      & P            & R         & X           & X         & D-3     \\ \hline
		\begin{tabular}[c]{@{}l@{}}Los estudiantes\\ pueden ver las\\ condiciones\\ necesarias para\\ completar una\\ actividad\end{tabular} & P            & R         & X           & X         &      \\ \hline
		\begin{tabular}[c]{@{}l@{}}Todas las\\ actividades tienen\\ la misma nota\\ máxima en el\\ calificador\end{tabular}                  & P            & R         & X           & X         &      \\ \hline
		\begin{tabular}[c]{@{}l@{}}Las preguntas de\\ los cuestionarios\\ tienen\\ retroalimentación\end{tabular}                                        & P           & R         & X           & X         &      \\ \hline
		\begin{tabular}[c]{@{}l@{}}Las preguntas de\\ opción multiple\\ puntúan de\\ forma justa\end{tabular}                                        & P           & R         & X           & X         &      \\ \hline
	\end{tabular}%
}
\end{table}

\subsection{Implementación:}
\begin{table}[H]
\resizebox{\textwidth}{!}{%
	\begin{tabular}{|l|l|l|l|l|l|}
		\hline
		Consulta                                                                                                   & Perspectivas & Diseñador & Facilitador & Proveedor & Proceso \\ \hline
		\begin{tabular}[c]{@{}l@{}}Los recursos\\ están actualizados\end{tabular}                                  & PT           & R         & X           & X         & I-1     \\ \hline
		\begin{tabular}[c]{@{}l@{}}Fechas de apertura\\ y cierre de tareas\\ son correctas\end{tabular}            & PT           & X         & R           & X         & R-2     \\ \hline
		\begin{tabular}[c]{@{}l@{}}Se detallan los\\ criterios de\\ evaluación\\ (rúbricas, ejemplos)\end{tabular} & PT           & R         & X           & X         & R-3     \\ \hline
		\begin{tabular}[c]{@{}l@{}}El calificador no\\ tiene demasiado\\ anidamiento\end{tabular} & PE           & R         & X           & X         &      \\ \hline
		\begin{tabular}[c]{@{}l@{}}Los alumnos\\ están divididos\\ en grupos\end{tabular}                          & TE           & X         &             & R         & I-6     \\ \hline
	\end{tabular}%
}
\end{table}

\subsection{Realización:}

\begin{table}[H]
\resizebox{\textwidth}{!}{%
	\begin{tabular}{|l|l|l|l|l|l|}
		\hline
		Consulta                                                                                                                                                & Perspectivas & Diseñador & Facilitador & Proveedor & Proceso \\ \hline
		\begin{tabular}[c]{@{}l@{}}El profesor\\ responde en los\\ foros dentro del\\ límite de 48 horas\\ lectivas desde que\\ se plantea la duda\end{tabular} & PT           & X         & R           & X         & R-2     \\ \hline
		\begin{tabular}[c]{@{}l@{}}Se ofrece\\ retroalimentación\\ de las tareas\end{tabular}                                                                   & PT           & X         & R           & X         & R-2     \\ \hline
		\begin{tabular}[c]{@{}l@{}}Las tareas están\\ calificadas\end{tabular}                                                                                  & PT           & X         & R           & X         & R-2     \\ \hline
		\begin{tabular}[c]{@{}l@{}}El calificador\\ muestra cómo\\ ponderan las\\ diferentes tareas\end{tabular}                                                & PT           & X         & R           & X         & R-2     \\ \hline
		\begin{tabular}[c]{@{}l@{}}El tiempo de\\ los cuestionarios\\ está bien ajustado\end{tabular}                                        & P           & R         & X           & X         &      \\ \hline
		\begin{tabular}[c]{@{}l@{}}Los cuestionarios\\ tienen una dificultad\\ razonable\end{tabular}                                        & P           & R         & X           & X         &      \\ \hline
		\begin{tabular}[c]{@{}l@{}}Las preguntas de\\ los cuestionarios\\ son buenas\\ discriminando\end{tabular}                                        & P           & R         & X           & X         &      \\ \hline
	\end{tabular}%
}
\end{table}

\subsection{Evaluación:}
\begin{table}[H]
\resizebox{\textwidth}{!}{%
	\begin{tabular}{|l|l|l|l|l|l|}
		\hline
		Consulta                                                                                    & Perspectivas & Diseñador & Facilitador & Proveedor & Proceso \\ \hline
		\begin{tabular}[c]{@{}l@{}}La mayoría de\\ alumnos responden\\ a los feedbacks\end{tabular} & PTE          & X         & X           & R         & E-2     \\ \hline
		\begin{tabular}[c]{@{}l@{}}Se utilizan encuestas\\ de opinión\end{tabular}                  & PTE          & X         & X           & R         & E-2     \\ \hline
	\end{tabular}%
}
\end{table}


\capitulo{3}{Conceptos teóricos}


\section{Definiciones}

\subsection{E-learning}

El e-learning es la enseñanza impartida por medios electrónicos como internet, plataformas virtuales, medios audiovisuales…etc.

\subsection{Moodle}

Moodle es una plataforma de aprendizaje que permite a los profesores crear entornos de aprendizaje altamente personalizables. Fue creado por Martin Dougiamas que publicó su primera versión el 20 de agosto de 2002\cite{dougiamas2002interpretive}.

\subsection{Calidad}

La calidad se puede definir como la capacidad de satisfacer una serie de necesidades y en el caso del e-learning se trata de las necesidades educativas del alumno, como la calidad del material educativo o la ayuda a la comprensión de este.

\subsection{Web scraping}

Es el conjunto de técnicas utilizadas para extraer y almacenar información de la web, un programa que hace esto de forma automática se llama web crawler.

\subsection{Web services}

Es un medio de comunicación entre dos aplicaciones en ordenadores distintos dentro de una red mediante el uso de distintos protocolos. Se puede pedir cierta información o acciones al servidor por medio de llamadas a funciones. En el caso de los web services que proporciona Moodle existe un servidor que hace de intermediario entre cliente y proveedor \cite{moodle-2020}.

Ahora paso a explicar las tres dimensiones del marco de referencia de calidad del e-learning del MOOQ \cite{stracke2018quality}, en el que me baso:
\subsection{Fases}

\subsubsection{Análisis}
En esta fase se definen los objetivos, el contexto, y los recursos (docentes, tiempo, presupuesto...) para la ejecución para comprender la situación inicial.

\subsubsection{Diseño}
En esta fase se define lo que se planea hacer a partir de los resultados de la fase de análisis, como por ejemplo el enfoque de enseñanza que se piensa llevar a cabo.

\subsubsection{Implementación}
En esta fase se define de qué manera se van a llevar a cabo los planes descritos en la fase de diseño, como por ejemplo de qué manera se va a producir el contenido.

\subsubsection{Realización}
Esta fase es en la que se interactúa con el alumno, se gestionan los problemas técnicos y las dudas de los alumnos, además de evaluar su aprendizaje.

\subsubsection{Evaluación}
En esta última fase se evalúa la calidad del resto de fases mediante encuestas, entrevistas, u otros medios.

\subsection{Roles}

\subsubsection{Diseñadores}
Los encargados de decidir de qué forma se van a impartir el curso y generan el contenido (Autores, expertos en el tema, diseñadores instruccionales). 

\subsubsection{Facilitadores}
Son aquellos que conocen la materia a enseñar y son capaces de explicarlo y dar feedback, además de seguir el aprendizaje de los alumnos.

\subsubsection{Proveedores}
Son los encargados de proveer los medios digitales para llevar a cabo el e-learning (programadores, diseñadores y desarrolladores de software).

\subsection{Perspectivas}

\subsubsection{Pedagógica}
El punto de vista que se centra en el contenido y en el aprendizaje por parte del alumno. Los procesos relacionados tienen que ver con el contenido, el feedback, y la evaluación de los alumnos.

\subsubsection{Tecnológica}
El punto de vista que se centra en las necesidades tecnológicas del curso. La mayoría de los procesos está relacionada con esta perspectiva debido a la naturaleza del e-learning.

\subsubsection{Estratégica}
El punto de vista que se centra en la consecución de los objetivos del curso dentro del tiempo y presupuesto establecidos para este. Los procesos relacionados tienen que ver con los objetivos, conceptos y el contexto en el que se enseña (presupuesto, demanda, competencia...).


\section{Consultas}
En este apartado definiré qué información intentamos conseguir de cada fase (nos vamos a centrar en las fases de diseño, implementación y realización), rol y perspectiva mediante distintos métodos. La mayoría de consultas provienen de la lista de comprobación de UBUCEV de asignaturas virtuales.
Leyenda:
Responsabilidad: R=Responsable,X=Involucrado
Perspectivas: P=Pedagógica, T=Tecnológica, E=Estratégica

\subsection{Diseño:}
\begin{table}[H]
	\resizebox{\textwidth}{!}{%
		\begin{tabular}{|l|l|l|l|l|l|l|l|}
			\hline
			Consulta                                                                                               & Queries                                                                                                                      & Perspectivas & Diseñador & Facilitador & Proveedor & Proceso & \begin{tabular}[c]{@{}l@{}}Elementos\\ examinados\end{tabular}              \\ \hline
			\begin{tabular}[c]{@{}l@{}}Las opciones de\\ progreso del \\ estudiante\\ están activadas\end{tabular} & \begin{tabular}[c]{@{}l@{}}Llamar a la función \\ core\_completion\_ge-\\ t\_activities\_comple-\\ tion\_status\end{tabular} & P            & R         & X           &           & D-5     & Temas                                                                       \\ \hline
			\begin{tabular}[c]{@{}l@{}}Se proporcionan\\ contenidos \\ en diferentes \\ formatos\end{tabular}      & \begin{tabular}[c]{@{}l@{}}Comprobar mediante\\ webscraping que hay\\ contenido audiovisual\\ embedido\end{tabular}          & PT           & R         & X           & X         & D-4     & \begin{tabular}[c]{@{}l@{}}Temas, \\ tareas y \\ cuestionarios\end{tabular} \\ \hline
		\end{tabular}%
	}
\end{table}

\subsection{Implementación:}
\begin{table}[H]
	\resizebox{\textwidth}{!}{%
		\begin{tabular}{|l|l|l|l|l|l|l|l|}
			\hline
			Consulta                                                                     & Queries                                                                                                                         & Perspectivas & Diseñador & Facilitador & Proveedor & Proceso & \begin{tabular}[c]{@{}l@{}}Elementos\\ examinados\end{tabular} \\ \hline
			\begin{tabular}[c]{@{}l@{}}Los recursos \\ están actualizados\end{tabular}   & \begin{tabular}[c]{@{}l@{}}Accede a las fechas\\ de los archivos para\\ comprobar si son \\ recientes o no\end{tabular}         & PT           & R         & X           & X         & I-1     & Archivos                                                       \\ \hline
			\begin{tabular}[c]{@{}l@{}}La división en \\ grupos es correcta\end{tabular} & \begin{tabular}[c]{@{}l@{}}Comprobar en los logs\\ que no hay ningún\\ alumno que no\\ pertenezca a ningún\\ grupo\end{tabular} & TE           & X         &             & R         & I-4     & Alumnos                                                        \\ \hline
		\end{tabular}%
	}
\end{table}

\subsection{Realización:}

\begin{table}[H]
	\resizebox{\textwidth}{!}{%
		\begin{tabular}{|l|l|l|l|l|l|l|l|}
			\hline
			Consulta                                                                                                                                              & Queries                                                                                                                                                  & Perspectivas & Diseñador & Facilitador & Proveedor & Proceso & \begin{tabular}[c]{@{}l@{}}Elementos\\ examinados\end{tabular}   \\ \hline
			\begin{tabular}[c]{@{}l@{}}Fechas de apertura\\ y cierre de tareas\\ son correctas\end{tabular}                                                       & \begin{tabular}[c]{@{}l@{}}Comprobar que las\\ fechas son de hace\\ menos de 6 meses y\\ que la de apertura es\\ anterior a la de cierre\end{tabular}    & PT           & X         & R           & X         & R-2     & \begin{tabular}[c]{@{}l@{}}Tareas y\\ cuestionarios\end{tabular} \\ \hline
			\begin{tabular}[c]{@{}l@{}}Se detallan los\\ criterios de\\ evaluación \\ (rúbricas, ejemplos)\end{tabular}                                           & \begin{tabular}[c]{@{}l@{}}Comprobar mediante\\ webscraping que hay\\ una rúbrica en la tarea\end{tabular}                                               & PT           & R         & X           & X         & R-3     & Tareas                                                           \\ \hline
			\begin{tabular}[c]{@{}l@{}}El profesor responde\\ en los foros dentro\\ del límite de 48 horas\\ lectivas desde que se\\ plantea la duda\end{tabular} & \begin{tabular}[c]{@{}l@{}}Comprobación de la\\ actividad en los foros\\ en los logs\end{tabular}                                                        & PT           & X         & R           & X         & R-2     & Foros                                                            \\ \hline
			\begin{tabular}[c]{@{}l@{}}Se ofrece\\ retroalimentación\\ de las tareas\end{tabular}                                                                 & \begin{tabular}[c]{@{}l@{}}Comprobación en los\\ logs de que haya un\\ comentario del\\ profesor en las tareas\\ cerradas hace\\ una semana\end{tabular} & PT           & X         & R           & X         & R-2     & Tareas                                                           \\ \hline
			\begin{tabular}[c]{@{}l@{}}Las tareas están\\ calificadas\end{tabular}                                                                                & \begin{tabular}[c]{@{}l@{}}Comprobación en los\\ logs de que se\\ hayan calificado\\ todas las tareas\\ cerradas hace una\\ semana\end{tabular}          & PT           & X         & R           & X         & R-2     & Tareas                                                           \\ \hline
			\begin{tabular}[c]{@{}l@{}}El calificador\\ muestra cómo\\ ponderan las\\ diferentes tareas\end{tabular}                                              & \begin{tabular}[c]{@{}l@{}}Comprobar mediante\\ webscraping que\\ aparece la columna\\ "Peso calculado" en\\ el calificador\end{tabular}                 & PT           & X         & R           & X         & R-2     & Calificador                                                      \\ \hline
		\end{tabular}%
	}
\end{table}

\subsection{Evaluación:}
\begin{table}[H]
	\resizebox{\textwidth}{!}{%
		\begin{tabular}{|l|l|l|l|l|l|l|l|}
			\hline
			Consulta                                                                                    & Queries                                                                                                      & Perspectivas & Diseñador & Facilitador & Proveedor & Proceso & \begin{tabular}[c]{@{}l@{}}Elementos\\ examinados\end{tabular} \\ \hline
			\begin{tabular}[c]{@{}l@{}}La mayoría de\\ alumnos responden\\ a los feedbacks\end{tabular} & \begin{tabular}[c]{@{}l@{}}Llamar a la función\\ mod\_feedback\_ge-\\ t\_non\_respondents\end{tabular}       & PTE          & X         & X           & R         & E-2     & Feedbacks                                                      \\ \hline
			\begin{tabular}[c]{@{}l@{}}Se utilizan\\ encuestas de\\ opinión\end{tabular}                & \begin{tabular}[c]{@{}l@{}}Llamar a la función\\ mod\_survey\_ge-\\ t\_surveys\_by\_cour-\\ ses\end{tabular} & PTE          & X         & X           & R         & E-2     & Encuestas                                                      \\ \hline
		\end{tabular}%
	}
\end{table}


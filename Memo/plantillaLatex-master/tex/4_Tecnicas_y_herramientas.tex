\capitulo{4}{Técnicas y herramientas}
\section{Desarrollo ágil}
En este proyecto se ha llevado a cabo un desarrollo ágil, llevando a cabo diversos procesos de este como el desarrollo iterativo o la implementación continua.
\subsection{Desarrollo iterativo}
El desarrollo consiste en la revisión cíclica sobre un mismo trabajo.
He dividido el trabajo en sprints de una duración de 14 días cada uno, teniendo reuniones con los tutores entre sprint y sprint, esto lo vuelve iterativo respecto al tiempo, pero también ha habido que hacer revisiones a ciertos aspectos de la aplicación como la lista de aspectos a mejorar.
\subsection{Desarrollo incremental}
El desarrollo incremental consiste en crear un prototipo funcional e ir añadiendo nuevas funcionalidades tras recibir la recibir la retroalimentación del cliente. Está fuertemente relacionado con el desarrollo iterativo y es difícil separar los dos conceptos. En bastantes de los sprints se hacen ``incrementos verticales'' que son mejoras que puede percibir el usuario en la aplicación. \imagen{Incremental.png}{Desarrollo incremental por sprints}
Los sprints se pueden ver en \url{https://github.com/RobertoArastiBlanco/eLearningQA/milestones}.
\subsection{Control de versiones}
El control de versiones es la utilización de herramientas que registran las modificaciones en el código de un proyecto, permitiendo acceder a las diferentes versiones. Estas herramientas permiten el trabajo concurrente de varios desarrolladores y permiten la aplicación de más procesos ágiles como la implementación continua.
\subsubsection{GitHub}
Github es un sitio de hospedaje de repositorios de código gratis para repositorios públicos.
GitHub Actions, integrado en GitHub, permite definir flujos de trabajo y lo he utilizado para implementar la integración continua.
\subsubsection{GitHub Desktop}
GitHub Desktop es una aplicación que proporciona una interfaz gráfica para interactuar con tus repositorios en GitHub. Simplifica las acciones más comunes asociándoles botones, muestra todas las diferencias introducidas antes de realizar un commit, gestiona tus credenciales de forma automática, y permite alternar entre ramas y repositorios en dos clicks, con lo que se agiliza mucho el control de versiones.
\subsection{Publicaciones frecuentes}
Una publicación consiste en dejar el producto a disposición de los usuarios para recibir retroalimentación ya sean críticas o halagos. Durante el desarrollo del proyecto he publicado cuatro versiones diferentes de la aplicación, desde la v0.1 hasta la v0.4. \imagen{Releases.png}{Versiones publicadas de la publicación}
Las versiones se pueden ver en \url{https://github.com/RobertoArastiBlanco/eLearningQA/tags}.
\subsection{Refactorización}
Refactorizar consiste en mejorar el código en su estructura sin alterar el comportamiento externo. He realizado dos grandes refactorizaciones de código al proyecto durante el desarrollo, la primera fue a mediados de enero para refactorizar los test y que utilicen archivos JSON, la segunda fue a principios de febrero con el objetivo de arreglar la mayoría de errores descubiertos gracias a la implantación de SonarCloud.
\subsection{Uso de test unitarios}
Un test unitario es un fragmento corto de código que ejecuta parte del código fuente de un programa y comprueba el resultado devuelto para considerar si el comportamiento del programa coincide con lo esperado. He utilizado JUnit para la ejecución de los test.
\subsubsection{JUnit}
JUnit es un framework para la realización de tests en Java. Es compatible con varios entornos de desarrollo integrados e incluso se puede usar mediante linea de comandos.
\subsection{Construcción automática}
La construcción automática es el uso de herramientas como Maven o Gradle para automatizar procesos como la compilación, la ejecución de tests y el empaquetado del software.
\subsubsection{Maven}
Maven es una herramienta para la construcción de proyectos software. Utiliza un archivo definido dentro del proyecto llamado ``pom.xml'' para definir la configuración necesaria para construir el proyecto como las dependencias o el formato al que compilar.
\subsection{Integración continua}
La integración continua consiste en el componente de automatización en el repositorio de código y tiene como prerrequisito la construcción automática.
El proyecto contiene un archivo yml que especifica las acciones a seguir para la construcción automática del proyecto y el análisis de calidad de SonarCloud tras cada \textit{push} al repositorio.
\subsection{Despliegue continuo}
El despliegue continuo consiste en automatizar el despliegue de la aplicación aprovechando la infraestructura de integración continua ya establecida en el repositorio. Los cambios se producen en la aplicación en vivo. He desplegado mi aplicación usando Heroku. El despliegue de la aplicación se encuentra en \url{https://elearningqa.herokuapp.com}.
\subsubsection{Heroku}
Heroku es una plataforma de hospedaje y computación de aplicaciones web dinámicas. Tiene una versión para hospedar una aplicación no comercial y poder ejecutarla durante 500 horas al mes de forma gratuita.

\subsection{Herramienta de calidad de código: SonarCloud}
SonarCloud es un servicio en la nube de análisis de código que detecta code smells, bugs, y vulnerabilidades de seguridad. Lo tengo integrado en el ciclo de integración continua. Permite definir un ``Quality gate'' para que la integración continua falle en caso de no cumplirse alguna de las condiciones definidas sobre la calidad del código.
También, al ser SonarCloud una herramienta de control de calidad, ha servido de inspiración al implementar la lista de aspectos a mejorar incluida en los informes que genera la aplicación.
Se puede ver la evolución de la calidad del código en la figura \ref{fig:sonarcloud} y en \url{https://sonarcloud.io/project/activity?category=QUALITY_GATE&id=RobertoArastiBlanco_MoodleQA}.

\section{Herramientas de desarrollo}
\subsection{Entorno de desarrollo integrado: IntelliJ IDEA}
Para el desarrollo de la aplicación web en Java, he elegido IDEA en oposición a Eclipse.
A pesar de que conozco Eclipse desde hace tres años y esta es la primera vez que utilizo IDEA, el año pasado utilicé Android Studio para desarrollar una aplicación Android, durante ese desarrollo me dí cuenta de que estaba basado en IDEA y que me sentía mucho más cómodo programando en Java en IDEs de JetBrains que en Eclipse. Facilita bastante la refactorización y escritura del código debido a funciones como el completado de código y las acciones rápidas, funcionalidades que también posee Eclipse, pero que en IDEA son más intuitivas y satisfactorias.
\subsection{Paso de JSON a POJO: Json2CSharp.com}
Json2CSharp.com es una página que permite la conversión de un objeto JSON a su POJO en Java correspondiente entre otras. La he utilizado en el proceso de implementación de las comprobaciones sobre los cursos Moodle debido a que las respuestas a las llamadas REST tenían que ser deserializadas a objetos para poder manejarlas con facilidad.
\subsection{Redacción en HTML: HtmlNotepad}
HtmlNotepad es un editor HTML que facilita la redacción de textos en HTML. Contiene gran cantidad de atajos de teclado que agiliza mucho el formateo de texto. Lo he utilizado para la redacción de la página del manual de usuario.
\subsection{Framework CSS: Bootstrap}
Bootstrap es un conjunto de librerías de CSS y javascript de código abierto creado por empleados de Twitter. Ofrece cantidad de recursos que facilitan la disposición de elementos en pantalla y contiene elementos como acordeones y carruseles ya implementados.
\subsection{Librería de generación de gráficos: Plotly}
Plotly es una librería de generación de gráficos para elm lenguaje de programación Python que también tiene una versión para Javascript. La he elegido por estar recomendada por W3Schools, una pagina web de tutoriales de desarrollo web creada en 1998, y por ser capaz de dibujar líneas independientes entre si al usar pares de coordenadas x e y y no tablas de valores.



\section{Herramientas de documentación}
\subsection{Redacción de memoria y anexos: TeXstudio}
TeXstudio es un editor de LaTeX. Contiene cantidad de herramientas necesarias para la creación de documentos en LaTeX y cuenta con un visor de PDF.
\subsection{Generación de tablas: TablesGenerator.com}
TablesGenerator.com es una página que permite crear una tabla con facilidad y puede convertirla a distintos formatos (LaTeX, HTML, texto simple...). La he utilizado para diseñar las tablas utilizadas en la aplicación y las tablas en memoria y anexos.
\subsection{Generación de diagramas UML: ArgoUML}
ArgoUML es una herramienta de dibujo de diagramas UML. Lo he utilizado para todos los diagramas de los anexos. La he utilizado para realizar diagramas de clases, de secuencia, y de casos de uso. Aceleró bastante el tiempo de modelado pero es bastante ortopédico en el sentido que hay ligeros cambios que no permite realizar.
\imagen{DiagramaSecuencia.png}{Diagrama de secuencia generado mediante ArgoUML}

\section{Patrón de diseño: Fachada}
El patrón de diseño fachada consiste en crear una clase que haga de intermediario entre el cliente y uno o varios subsistemas de la aplicación con varios propósitos: Simplificar y centralizar el control, actuar como elemento de seguridad restringiendo el acceso a ciertas partes, y separar responsabilidades de los subsistemas. Un mismo sistema podría tener varias fachadas distintas que den un mismo servicio de distintas formas, por ejemplo, en este caso, la fachada se utiliza para generar parte del contenido de la aplicación web, pero si quisiera trasladar la aplicación a una de escritorio solo se tendría que crear una nueva fachada dejando los sistemas subyacentes intactos \cite{gamma1995design}. Para más información sobre el uso concreto que se le da a este patrón de diseño, consultar el apartado C.4 de los anexos.
 
\section{Herramientas para acceder a la información}
A continuación se describen los medios que se han barajado para la extracción de la información de los cursos que queremos analizar con nuestra aplicación. Finalmente solo se han utilizado servicios web, sin embargo, es interesante ponderar sus ventajas.
\begin{itemize}
	\item \textbf{Web scraping:}
	Es el conjunto de técnicas utilizadas para extraer y almacenar información de la web, un programa que hace esto de forma automática se llama \textit{web crawler}. Algunas de estas técnicas nos permiten obtener información que no podríamos sacar con otras herramientas como por ejemplo si una página contiene enlaces externos a partir de su código HTML.
	\item \textbf{Web services:}
	Es un medio de comunicación entre dos aplicaciones en ordenadores distintos dentro de una red mediante el uso de distintos protocolos. Se puede pedir cierta información o acciones al servidor por medio de llamadas a funciones. En el caso de los \textit{web services} que proporciona Moodle existe un servidor que hace de intermediario entre cliente y proveedor \cite{moodle-2020}. Por ejemplo, llamando a la función \texttt{core\_grades\_get\_grades} podríamos obtener las notas de un alumno.
	\item \textbf{Logs:}
	Son registros de actividad que Moodle crea a partir de los eventos que realizan los usuarios, como publicar en un foro o empezar un cuestionario. Se pueden descargar en forma de archivo. Un ejemplo del uso que le daríamos sería obtener la actividad de un profesor para saber si responde a las dudas de los alumnos en los foros.
\end{itemize}
\section{Framework de desarrollo web}
Respecto a qué framework utilizar para el desarrollo de la aplicación web hemos barajado estas opciones:
\begin{itemize}
	\item \textbf{Spring:}
	Es un framework para la creación de aplicaciones Java. Gestiona las dependencias entre objetos de forma automática lo que permite un bajo nivel de acoplamiento y ofrece un framework para desarrollo de aplicaciones web. Su fuerte es la escalabillidad, pero es complicado aprender a usarlo.
	\item \textbf{Vaadin:}
	Es un framework de desarrollo web con una gran librería de componentes web. Usa GWT (\textit{Google Web Toolkit}) para compilar Java a JavaScript y evitar al programador usar otros lenguajes aparte de Java. Su versión de pago ofrece una herramienta de edición gráfica que acelera el proceso de creación. Su fuerte es la velocidad de creación de prototipos, pero tiene mala escalabilidad.
\end{itemize}
\imagen{SpringVsVaadin.png}{Conparativa en popularidad de Spring y Vaadin}

De entre estas dos nos decantamos por Spring, ya que la escalabilidad es bastante mejor y es mucho más popular que Vaadin, lo que nos ayudará a encontrar documentación y tutoriales a la hora de encontrar problemas durante el desarrollo.
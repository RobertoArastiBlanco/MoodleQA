\capitulo{5}{Aspectos relevantes del desarrollo del proyecto}
\section{Ciclo de vida}
La realización de este trabajo se ha llevado a cabo en sprints de una duración de 14 días con reuniones entre sprint y sprint, y con frecuencia han habido reuniones a mitad de estos.

El proyecto empezó sintetizando una lista de aspectos a comprobar en los cursos Moodle a partir del marco de referencia de calidad del e-learning de MOOQ \cite{stracke2018quality} y un documento interno de UBUCEV proporcionado por los tutores.
Después tuve que decidir si crear una aplicación de escritorio o una aplicación web y qué framework utilizar para desarrollar la aplicación.
A partir de ahí, se hizo un prototipo para comprobar que era capaz de acceder a los \textit{web services} de \textit{Moodle} desde mi aplicación. Este prototipo solo mostraba la lista de cursos accedidos recientemente a partir de unas credenciales para la página de demostración de Moodle llamada Mount Orange School.

Más tarde establecí el ciclo de integración continua/despliegue continuo y lancé la aplicación en Heroku. Una semana después, al final del sprint había creado una versión del informe específico que solo realizaba una comprobación sobre los cursos. A partir de ahí, debido a las fechas (segunda quincena de diciembre) no hubo ninguna reunión con los tutores hasta la vuelta de las vacaciones de navidad. Durante ese periodo, añadí el resto de las comprobaciones al informe y me dediqué a completar partes de la memoria. Decidimos retrasar la entrega del trabajo al segundo semestre.

El siguiente paso fue añadir \textit{SonarCloud} para analizar el código, y arreglar la gran mayoría de los errores encontrados. Se añadió \textit{Bootstrap} para mejorar bastante el estilo de la página. Se añadieron varias mejoras de características no funcionales a la aplicación: un logo, configuraciones intercambiables, alertas de las causas de los fallos de las comprobaciones, página de error y gráficos de evolución de la calidad de los cursos. El resto del desarrollo se dedicó a solucionar errores pequeños.

\section{Proceso de obtención de llamadas a los servicios web}
Para obtener la información necesaria para las comprobaciones sobre los cursos de Moodle he tenido que realizar llamadas REST a distintas funciones de la API de servicios web de Moodle \cite{wsapifunctions-2021}. Sin embargo, la tabla que detalla la lista de funciones de la API omite los parámetros necesarios para las llamadas a las funciones.
Para averiguar qué parametros debía utilizar en cada función tuve que acceder al repositorio de Moodle \cite{moodlerepository-2022} y encontrar las funciones que detallaban el funcionamiento de la función de la API en la que estaba interesado cada vez.

El nombre de una función se puede dividir en dos partes: el nombre del componente que posee la función, y el nombre de la función. Las funciones están escritas en PHP y aparecen dentro de los archivos llamados ``externallib.php''. Para cada función existen tres funciones asociadas: 
\begin{itemize}
	\item <NOMBRE DE LA FUNCIÓN>
	\item <NOMBRE DE LA FUNCIÓN>\_parameters
	\item <NOMBRE DE LA FUNCIÓN>\_returns
\end{itemize}
La función a secas define el comportamiento de la función, y en su declaración se puede ver qué parámetros son opcionales y cuáles no.
La función con ``parameters'' al final describe los nombres y los tipos de datos de los parámetros que espera recibir la función.
La función con ``returns'' al final describe los nombres y los tipos de datos de los atributos que devuelve la función.
Por ejemplo, si quiero averiguar como llamar a la función mod\_forum\_get\_forums\_by\_courses
tengo acceder al fichero /mod/forum/externallib.php (supuesto por el nombre del componente que contiene la función, ``mod\_forum'' en este caso) y buscar las funciones get\_forums\_by\_courses, get\_forums\_by\_courses\_parameters, y get\_forums\_by\_courses\_returns.

Para probar el funcionamiento de las llamadas y al mismo tiempo obtener las clases que debía definir como POJO para manejar los datos recibidos como JSON (JavaScript Object Notation), es una forma de representar un objeto en formato de texto muy usado para transmitir información en aplicaciones web, hice lo siguiente:
Uso la página de demostración de Moodle llamada Mount Orange School para hacer pruebas.
Hago llamadas REST de forma manual con el navegador Chrome y obtengo el token con las credenciales de profesor (usuario:``teacher'' contraseña:``moodle'').\imagen{Token.png}{Obtención del token} Después utilizo el token para la siguiente llamada y obtengo una respuesta en formato JSON. En este caso la llamada solo necesita el nombre de la función, el formato de respuesta, y el token, pero en la mayoría de casos se requieren parámetros como identificadores de foro o de actividad. \imagen{JSON.png}{Obtención del JSON} Copio la respuesta en un conversor de JSON a POJO.\imagen{Conversor.png}{Uso del conversor} Creo las clases correspondientes, añado un constructor vacío y encapsulo los atributos. 


\section{Integración continua y despliegue continuo}
La integración continua consiste en la automatización de la compilación y ejecución de pruebas cada vez que se suben cambios al repositorio.
El despliegue continuo es la automatización del despliegue de un producto tras cada cambio en el repositorio.
He implementado la integración continua del proyecto con GitHub Actions, primero, establecí en el archivo pom.xml que la compilación del proyecto fuera en formato WAR (Web Application Resource), luego, creé el archivo ``maven.yml'' en la carpeta de workflows para establecer que cada vez que se realice un push en la rama principal el proyecto se compile y se ejecuten los tests con Maven.

He implementado el despliegue continuo en Heroku, la mayoría del proceso ha sido bastante intuitiva, ya que una de las opciones que ofrecía era GitHub como método de despliegue pero debido a que mi repositorio contiene por un lado la memoria y por otro el proyecto software, he tenido que añadir un buildpack (conjuntos de scripts de código abierto usados para compilar las aplicaciones en Heroku) que permite especificar una subcarpeta del repositorio para usarla como directorio raíz del proyecto software. También establecí en las opciones que el despliegue automático espere a que se supere la integración continua. También, tras definir una quality gate en SonarCloud, ha pasado a formar parte de la integración continua, y, en caso de no cumplirse los estándares de la quality gate (figura \ref{fig:qualitygate}), la integración continua cuenta como fallida.

\imagen{QualityGate.png}{\label{fig:qualitygate}Quality gate usada en el proyecto}

\section{Fallo de seguridad Heroku/TravisCI/Github}
El 9 de abril de 2022 unos repositorios privados de Github fueron descargados de forma indebida debido a un ataque por medio de Heroku y TravisCI. El atacante robó tokens de autorización de Github almacenados por Heroku y TravisCI el día 7 y los utilizó para buscar repositorios basándose en las organizaciones a las que pertenecían el día 8 para acabar descargándolos el día 9. Github descubrió esto el 12 de abril y avisó a Heroku el día siguiente \cite{githubsecurity-2022}. Tras tres días de investigación, Heroku revocó los tokens el 16 de abril, desabilitando el despliegue automático en el proceso y afirmó no reestablecerlo hasta cerciorarse de que hacerlo sea seguro. Heroku reestableció el servicio el 25 de mayo \cite{herokusecurity-2022}.

\section{Uso del nombre Moodle}
Como el objetivo inicial del proyecto es integrar Moodle en la aplicación, el nombre de tanto el proyecto como la aplicación iba a ser ``MoodleQA'', pero en medio del desarrollo los tutores y yo descubrimos que al ser ``Moodle'' una marca registrada \cite{moodletrademark-2022}, existe cierta cantidad de restricciones sobre el uso de esta palabra. Las restricciones que nos conciernen son no poder utilizar ``Moodle'' en el nombre de tu software ni en el nombre de dominio ni usarlo para describir tu aplicación de forma que los usuarios crean que estás asociado con Moodle si no lo estás. Por ello he tenido que revisar todas las menciones que hago a Moodle tanto en la memoria y anexos como en la aplicación.

\section{Informes: pestañas vs breadcrumbs}
A la hora de mostrar los informes se presenta la disyuntiva de si hacerlo en una sola pestaña de navegador y permitir al usuario navegar entre los distintos cursos con ayuda de una miga de pan o breadcrumb, o por el contrario generar cada informe en una nueva pestaña. Me he decantado por generar una pestaña por informe por dos razones: el rendimiento y la comparación de informes. Cada vez que se genera un informe se descarga la información necesaria para las consultas y se realizan las consultas sobre dicha información, para un servidor puede que la carga no sea mucha y se puedan generar los informes al instante, pero ejecutando la aplicación en local desde mi ordenador portátil he llegado a tener esperas de alrededor de diez segundos por un solo informe. A la hora de comparar informes es más ágil el cambiar de pestaña que retroceder a la página anterior e ir al otro informe.

\section{SonarCloud}
En esta sección agrupo y expongo dos sucesos interesantes ocurridos relacionados con \textit{SonarCloud} tras la implantación de esta herramienta al repositorio.
\subsection{Definición del quality gate de SonarCloud}
En SonarCloud la quality gate es una serie de condiciones sobre la calidad del codigo, por ejemplo, que la puntuación de seguridad no esté por debajo de ``A''. Existe una quality gate predefinida llamada Sonar way, pero está definida para proyectos reales, así que en mi caso he decidido definir mi propia quality gate a partir de otros proyectos en Sonarcloud de la misma índole. Realicé una búsqueda y encontré 25 TFGs de la UBU, y calculé las medianas de dichos trabajos para obtener las condiciones de mi quality gate.

\subsection{SonarCloud en el ciclo de desarrollo}
Desde que añadí \textit{SonarCloud} al ciclo de integración y despliegue continuo he creado el hábito de comprobar los bugs, smells, y vulnerabilidades generados tras cada commit, esto evita que la cantidad de problemas se acumule y a su vez hace que el código sea más fácil de mantener. Además, el hecho de entender los code smells conforme los solucionaba me ha enseñado a tenerlos con menor frecuencia. A continuación se muestra un gráfico generado por SonarCloud del número de problemas a lo largo del tiempo en el que se aprecia que tras solucionar los problemas iniciales el gráfico se mantiene relativamente plano. \imagen{GraficoSonarCloud.png}{\label{fig:sonarcloud}Gráfico del número de problemas a lo largo del tiempo}

\section{Diseño del logo de la aplicación}
Debido a que la aplicación realiza una serie de consultas se me ocurrió que el logo lo formasen unas aspas y un check, pero sin el uso de colores el símbolo resultante puede resultar confuso, así que rodeé la parte del check con color verde y la parte de las aspas con color rojo para hacerlo más entendible.
Los colores usados son los de Bootstrap para crear concordancia con el resto de la página y de paso crear las imágenes de check y aspas usados en los resultados de las consultas y que de esta forma se mimeticen en el color de fondo de la celda. El texto del logo es blanco con reborde negro para hacerlo legible con independencia del color del fondo. \imagen{FullLogo.png}{Logo de la aplicación}

\section{Almacenamiento de registros de informes}
La aplicación almacena los registros de los informes generados para poder generar los gráficos de evolución. Ejecutando la aplicación en local no supone ningún problema, sin embargo, el almacenamiento en Heroku es volátil \cite{ephemeralHeroku-2018}. La solución a esto sería contratar un proveedor de active storage como Amazon S3 para tener un almacenamiento que funcione independiente a la aplicación \cite{activeStorageOverview-2019}.

\section{Redefinición de regla de comentarios del profesor}
Originalmente lo que se comprobaba era si el ratio entre items de calificación comentados dividido entre el número total de items comentables se encontraba por encima de cierto umbral, sin embargo, esto da falsos negativos en caso de que los estudiantes no realicen las entregas correspondientes. Si reducimos el concepto de items comentables a aquellos que tengan calificación evitaremos este problema.

También existe otro problema, actualmente, para poder realizar la consulta se comprueban los comentarios de la columna de comentarios de los calificadores de todos los alumnos, columna que el profesor puede elegir no mostrar en el calificador. Este problema causa que nuestra regla compruebe de forma implícita una decisión de diseño aparte de su objetivo principal, que es comprobar que el profesor dé retroalimentación a los alumnos al corregir (que corresponde a la fase de realización).
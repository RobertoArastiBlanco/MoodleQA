\apendice{Plan de Proyecto Software}

\section{Introducción}
En este apéndice se trata sobre los costes y viabilidad del proyecto. Contiene una planificación temporal que hablará de la cantidad de tiempo estimada que va a costar realizar el proyecto. También un estudio de viabilidad dividido en dos partes: un estudio de viabilidad económica que tratará los costes y beneficios teóricos de la realización del proyecto en un escenario real a pesar de que el fin de este trabajo sea educativo, y un estudio de viabilidad legal que discutirá si hay o no problemas derivados de las licencias de las herramientas usadas y/o infracciones de copyright.
\section{Planificación temporal}
En esta sección se habla de la cantidad de tiempo estimada necesaria al principio del proyecto y la cantidad de tiempo que ha acabado llevando, detallando lo ocurrido durante el proceso.

La planificación del proyecto se ha llevado a cabo en sprints de una duración de dos semanas. Normalmente se realizó una reunión entre sprint y sprint en la que se discutían los resultados del sprint anterior y los objetivos para el siguiente. En GitHub, los sprints están definidos como milestones (hitos) y los objetivos del sprint como issues (problemas, asuntos). Para cada sprint he creado un nuevo milestone y he incluido los issues respectivos a los objetivos del sprint a dicho milestone.

Al comienzo estimé que para completar el proyecto necesitaría 400 horas de trabajo a lo largo de 16 semanas, lo que conllevaría 25 horas de trabajo a la semana.

Ahora paso a resumir los objetivos y sucesos de cada sprint:

\subsection{Sprint 0 (7/10/2021 - 21/10/2021)}
	En este sprint los objetivos propuestos eran instalar la plantilla de la memoria en LaTeX proporcionada por la UBU, redactar una introducción y definir los objetivos del proyecto, y sintetizar una lista de comprobaciones a realizar por parte de la aplicación a partir de unos marcos de calidad. En la reunión de cierre del sprint los tutores se mostraron disconformes con mi forma de redactar excesivamente técnica y breve y se propuso el realizar reuniones a mitad de sprint cuando yo lo viera conveniente.
\subsection{Sprint 1 (21/10/2021 - 4/11/2021)}
	En este sprint se definieron como objetivos incluir en la introducción un resumen de la situación actual respecto del control de calidad en el e-learning y añadir las dimensiones de roles y perspectivas a la lista de comprobaciones y añadir la lista a la memoria. En la reunión de mitad de sprint se remarcó la falta de referencias bibliográficas y se me pidió que expandiera el apartado de conceptos teóricos. En la reunión de cierre del sprint los tutores se mostraron bastante más conformes con mi trabajo.
\subsection{Sprint 2 (4/11/2021 - 18/11/2021)}
	En este sprint se definieron como objetivos añadir definiciones al apartado de conceptos teóricos, mostrar la lista de consultas definida en forma de tabla, y decantarme por alguna de los tres frameworks para el desarrollo de la aplicación (Vaadin, JavaFX, o Spring). En la reunión de mitad de sprint se me dieron indicaciones sobre aspectos a mejorar que realicé antes de acabar el sprint. Elegí Spring para crear la aplicación web.
\subsection{Sprint 3 (18/11/2021 - 2/12/2021)}
	En este sprint el objetivo principal fue la creación de un prototipo. Creé un prototipo muy simple que mostraba la lista de cursos recientemente accedidos en la plataforma de demostración Mount Orange School tras introducir usuario y contraseña. En la reunión de final de sprint se me propuso utilizar el patrón de diseño fachada para separar la responsabilidad de acceder a los servicios web y generar el contenido para mostrarlo en HTML.
\subsection{Sprint 4 (2/12/2021 - 16/12/2021)}
	En este sprint se definieron como objetivos comenzar con el diseño de la aplicación, crear un pipeline de integración continua/despliegue continuo con GitHub Actions y Heroku, y añadir funcionalidad al prototipo. En la primera semana del sprint creé un boceto de la interfaz de la aplicación y un diagrama de clases preliminar, además de implementar la integración continua con GitHub Actions y despliegue continuo en Heroku. En la reunión de mitad de sprint mostré los avances que había conseguido a Raúl Marticorena (Carlos López estaba ocupado ese día) y me instó a añadir funciones al prototipo. Durante la segunda semana añadí una versión reducida del informe específico que mostraba la comprobación de si un curso tenía grupos definidos o no. En la reunión de final de sprint mostré los avances que había conseguido a Carlos López (Raúl Marticorena estaba ocupado ese día) y establecimos que no íbamos a poder reunirnos hasta la vuelta de las vacaciones y me motivó diciendo que a partir de ese punto el resultado del proyecto dependía de la cantidad de trabajo que invirtiera en él y que gracias al trabajo que había realizado hasta ese punto el proyecto estaba encarrilado.
\subsection{Sprint 5 (16/12/2021 - 30/12/2021)}
	En este sprint se definieron como objetivos refactorizar la fachada de forma que se puedan realizar tests sobre ella, añadir tests, y seguir implementando consultas sobre los cursos Moodle en el prototipo. Al final de este sprint no hubo reunión por motivo de las fiestas navideñas.
\subsection{Sprint 6 (30/12/2021 - 13/1/2022)}
	En este sprint decidí implementar las comprobaciones restantes, rellenar algunos apartados de la memoria y cambiar el estilo de la página del informe para que hubiera un código de cinco colores para distintos niveles de gravedad. En la reunión, a pesar de haber hecho un gran avance, decidí retrasar la entrega de este trabajo al segundo semestre.
\subsection{Sprint 7 (13/1/2022 - 27/1/2022)}
	En este sprint refactoricé los test unitarios para que carguen los objetos necesarios desde una serie de archivos json en vez de conectarse a la página de demo de Moodle que es accesible por cualquiera y está sujeta a cambios. También se refactorizó la fachada para no tener métodos extremadamente largos y cambios varios. A la reunión de final de sprint solo pudo acudir Carlos López, y propuso añadir cinco consultas que comprobaban aspectos de los cuestionarios.
\subsection{Sprint 8 (27/1/2022 - 10/2/2022)}
	En este sprint integré la herramienta de calidad de código SonarCloud al repositorio, revisé la API de servicios web de Moodle para ver si eran factibles las consultas propuestas en la reunión anterior, creé un archivo html a modo de boceto para trabajar en el diseño del informe e hice un apartado de buenas prácticas en la docencia online. En la reunión mostré el boceto html, con el que se mostraron medianamente conformes. En la reunión también se desestimó el implementar las consultas propuestas y dejarlo como una línea futura de trabajo.
\subsection{Sprint 9 (10/2/2022 - 24/2/2022)}
	En este sprint empecé a solucionar los errores descubiertos por SonarCloud y modifiqué el estilo de la mayoría de mis páginas con el framework Bootstrap. En la reunión, hablando del logo acacabamos descubriendo que al ser ``Moodle'' una marca registrada debíamos hacer cambios como por ejemplo el nombre de dominio del despliegue en Heroku.
\subsection{Sprint 10 (24/2/2022 - 10/3/2022)} 
	En este sprint cambié con Bootstrap el estilo de la página de informe, cambié todos los nombres problemáticos y solucioné casi todos los errores de código. En la reunión se decidió seguir trabajando en el estilo del informe e implementar una forma de generar alertas para los errores detectados por las consultas.
\subsection{Sprint 11 (10/3/2022 - 24/3/2022)} 
	En este sprint añadí las alertas y diseñé el logo de la aplicación. En la reunión se comentaron errores del informe y se decidió añadir gráficos a la página de informe y hacer perfiles de configuración intercambiables para distintos tipos de curso.
\subsection{Sprint 12 (24/3/2022 - 7/4/2022)} 
	En este sprint arreglé algunas de las alertas que no funcionaban bien y añadí los perfiles de configuración, pero aplacé añadir los gráficos.
\subsection{Sprint 13 (7/4/2022 - 21/4/2022)} 
	En este sprint arregle un error encontrado por los profesores, añadí los gráficos y una página de error, pero el cambio necesario para hacer las comprobaciones flexibles tenía demasiada complejidad. Solo Carlos López pudo acudir a la reunión, así que se decidió convocar otra la semana siguiente. 
\subsection{Sprint 14 (21/4/2022 - 28/4/2022)} 
	En este sprint se arreglaron problemas estéticos y pequeños bugs. En la reunión dimos un repaso a todos los errores y mejoras pendientes de la página de informe. También se decidió empezar a centrarse en la memoria y anexos.
\subsection{Sprint 15 (28/4/2022 - 12/5/2022)} 
	En este sprint se arreglaron más errores y se hicieron correcciones en memoria y anexos. En la reunión se comentaron más errores y se decidió continuar con las mismas tareas.
\subsection{Sprint 16 (12/5/2022 - 26/5/2022)} 
	En este sprint se siguió haciendo lo mismo que en el sprint anterior, arreglar errores y añadir contenido a la memoria. En la reunión se decidió pasar a hacer sprints de una semana debido a la cercanía de la fecha limite y centrarse en completar los anexos.
\subsection{Sprint 17 (26/5/2022 - 2/6/2022)} 

\subsection{Sprint 18 (2/6/2022 - 9/6/2022)} 


\section{Estudio de viabilidad}

\subsection{Viabilidad económica}
Para hacer rentable el desarrollo de la aplicación teniendo en cuenta que es de código abierto se podría adoptar un modelo SaaS (Software as a Service) en el que los clientes paguen una suscripción al dueño del software y este a su vez se encargue del hosting y mantenimiento de la aplicación.
Para amortizar el coste de desarrollo de la aplicación se necesitaría ...
\subsection{Viabilidad legal}
\subsubsection{Licencias de software}
Para determinar la licencia software que va a utilizar la aplicación hay que tener en cuenta las licencias utilizadas por las dependencias que utiliza la aplicación.

\begin{table}[H]
	\resizebox{\textwidth}{!}{%
		\begin{tabular}{|c|c|c|}
			\hline
			\textbf{Software}   & \textbf{Descripción}                                                                                                         & \textbf{Licencia} \\ \hline
			Spring Framework    & Framework para aplicaciones web                                                                                              & Apache 2.0        \\ \hline
			Tomcat Embed Jasper & \begin{tabular}[c]{@{}c@{}}Implementación de Tomcat que incluye\\ Jasper, el parser de JSP de Tomcat\end{tabular}            & Apache 2.0        \\ \hline
			JUnit               & Framework para tests unitarios en Java                                                                                       & EPL               \\ \hline
			Apache Commons IO   & \begin{tabular}[c]{@{}c@{}}Librería de utilidades varias (usado en \\ traducción de imágenes a arrays de bytes)\end{tabular} & Apache 2.0        \\ \hline
			Apache Log4j        & Librería para registro de logs                                                                                               & Apache 2.0        \\ \hline
			Bootstrap           & \begin{tabular}[c]{@{}c@{}}Librerías CSS y JavaScript para \\ páginas web\end{tabular}                                       & MIT               \\ \hline
			Plotly.js           & \begin{tabular}[c]{@{}c@{}}Librería JavaScript de \\ generación de gráficos\end{tabular}                                     & MIT               \\ \hline
		\end{tabular}%
	}
\end{table}
La licencia pública de Eclipse (EPL) es compatible con estas licencias, a continuación se mencionan sus posibilidades y obligaciones.
\begin{itemize}
	\item\textbf{Permite:} uso, reproducción, distribución, modificación, uso comercial y uso de patentes.
	\item\textbf{Obliga a:} revelar la fuente y el autor, mantener la misma licencia al redistribuir el software, distribuir el software libre de regalías.
	\item\textbf{No permite:} responsabilizar al autor o contribuidores por posibles daños, utilizar marcas propiedad del autor para promoción o publicidad.
\end{itemize}
\subsubsection{Uso del nombre ''Moodle'' en la aplicación}
Debido a que Moodle es una marca registrada y no soy un partner registrado de Moodle, tengo que atenerme a una serie de restricciones con respecto al uso de la palabra ''Moodle'' \cite{moodletrademark-2022}:
\begin{itemize}
	\item No puedo usar logos de ''Moodle'' sin consentimiento escrito de Moodle.
	\item No puedo usar ''Moodle'' en el nombre de mi software, ni en el de mi dominio ni en el de mi empresa.
	\item No puedo usar ''Moodle'' en palabras clave relacionadas con la publicidad.
	\item No puedo usar ''Moodle'' para describir servicios alrededor de Moodle de forma que la gente piense que estoy asociado a Moodle cuando no es así.
\end{itemize}

Al enterarme de esto a mitad del desarrollo he tenido que cambiar a ''eLearningQA'' el nombre del proyecto, el del repositorio, el del despliegue en Heroku para que el dominio deje de contener ''Moodle'', y he aclarado en el readme del proyecto y en la memoria que no estoy asociado con Moodle.

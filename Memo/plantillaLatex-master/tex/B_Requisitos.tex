\apendice{Especificación de Requisitos}
\label{apendice:B}
\section{Introducción}
En este apéndice se obtienen los requisitos funcionales y no funcionales del software en cuestión a partir de los objetivos generales y expectativas que tenemos del proyecto. Un documento de especificación de requisitos sirve como un medio de comunicación entre todas las partes interesadas en el desarrollo del software.
\section{Objetivos generales}
	Los objetivos generales de este proyecto son los que se listan a continuación:
\begin{itemize}
	\item
	Poder realizar una serie de comprobaciones sobre los cursos de Moodle que faciliten la evaluación de la calidad de estos.
	\item
	Poder aplicar un framework de calidad del e-learning para evaluar los distintos aspectos que influyen en la calidad de los cursos.
	\item
	Poder generar informes con información concisa que permitan responder a los desafíos que empobrecen la calidad de los cursos.
\end{itemize}

\section{Catalogo de requisitos}
La lista de requisitos funcionales es la siguiente:
\begin{itemize}
	\item R-01: la aplicación permitirá registrarse al docente para acceder a la información sobre los cursos.
	\item R-02: la aplicación permitirá al profesor desconectarse.
	\item R-03: la aplicación mostrará un listado de los cursos en los que se encuentre el profesor.
	\item R-04: se deben poder generar informes que evalúen la calidad de los cursos en los que se encuentra el profesor.
	\item R-05: se deben poder generar informes globales que resuman los informes de todos los cursos en los que se encuentre el profesor.
\end{itemize}

\section{Especificación de requisitos}
A partir de los requisitos se obtiene el siguiente diagrama de casos de uso:
\imagen{DiagramaCasosUso.PNG}{Diagrama de casos de uso}



\apendice{Especificación de Requisitos}

\section{Introducción}

\section{Objetivos generales}

\section{Catalogo de requisitos}
La lista de requisitos funcionales es la siguiente:
\begin{itemize}
	\item R-01: la aplicación permitirá registrarse al docente para acceder a la información sobre los cursos.
	\item R-02: la aplicación permitirá al profesor desconectarse.
	\item R-03: la aplicación mostrará un listado de los cursos en los que se encuentre el profesor.
	\item R-04: se deben poder generar informes que evalúen la calidad de los cursos en los que se encuentra el profesor.
	\item R-05: se deben poder generar informes globales que resuman los informes de todos los cursos en los que se encuentre el profesor.
\end{itemize}

\section{Especificación de requisitos}
A partir de los requisitos se obtiene el siguiente diagrama de casos de uso:
\imagen{DiagramaCasosUso.PNG}{Diagrama de casos de uso}
En la siguiente tabla se recogen las entidades asociadas a cada una de las consultas para determinar parte de las clases necesarias para el proyecto.
\begin{table}[H]
	\resizebox{\textwidth}{!}{%
		\begin{tabular}{|l|l|}
			\hline
			\textbf{Consulta}                                                                                                                                        & \textbf{Entidades}        \\ \hline
			Las opciones de progreso del estudiante están activadas                                                                                         & Curso            \\ \hline
			Se proporcionan contenidos en diferentes formatos                                                                                               & Curso            \\ \hline
			El curso tiene grupos                                                                                                                           & Grupo, Curso     \\ \hline
			El curso tiene actividades grupales                                                                                                             & Actividad, Curso \\ \hline
			\begin{tabular}[c]{@{}l@{}}Los estudiantes pueden ver las condiciones\\ necesarias para completar una actividad\end{tabular}                    & Curso            \\ \hline
			Todas las actividades tienen la misma nota máxima en el calificador                                                                             & Curso            \\ \hline
			Los recursos están actualizados                                                                                                                 & Curso            \\ \hline
			Fechas de apertura y cierre de tareas son correctas                                                                                             & Actividad, Curso \\ \hline
			Se detallan los criterios de evaluación (rúbricas, ejemplos)                                                                                    & Curso            \\ \hline
			El calificador no tiene demasiado anidamiento                                                                                                   & Curso            \\ \hline
			Los alumnos están divididos en grupos                                                                                                           & Alumno, Grupo    \\ \hline
			\begin{tabular}[c]{@{}l@{}}El profesor responde en los foros dentro del límite de\\ 48 horas lectivas desde que se plantea la duda\end{tabular} & Curso            \\ \hline
			Se ofrece retroalimentación de las tareas                                                                                                       & Tarea, Curso     \\ \hline
			Las tareas están calificadas                                                                                                                    & Tarea, Curso     \\ \hline
			El calificador muestra cómo ponderan las diferentes tareas                                                                                      & Curso            \\ \hline
			La mayoría de alumnos responden a los feedbacks                                                                                                 & Feedback, Curso  \\ \hline
			Se utilizan encuestas de opinión                                                                                                                & Survey, Curso    \\ \hline
		\end{tabular}%
	}
\end{table}


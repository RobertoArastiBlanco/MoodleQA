\apendice{Especificación de Requisitos}
\label{apendice:B}
\section{Introducción}
En este apéndice se obtienen los requisitos funcionales y no funcionales del software en cuestión a partir de los objetivos generales y expectativas que tenemos del proyecto. Un documento de especificación de requisitos sirve como un medio de comunicación entre todas las partes interesadas en el desarrollo del software.
\section{Objetivos generales}
	Los objetivos generales de esta aplicación son los que se listan a continuación:
\begin{itemize}
	\item
	Poder realizar una serie de comprobaciones sobre los cursos de Moodle que faciliten la evaluación de la calidad de estos.
	\item
	Poder aplicar un framework de calidad del e-learning para evaluar los distintos aspectos que influyen en la calidad de los cursos.
	\item
	Poder generar informes con información concisa que permitan responder a los desafíos que empobrecen la calidad de los cursos.
\end{itemize}

\section{Catalogo de requisitos}
La lista de requisitos funcionales es la siguiente:
\begin{itemize}
	\item R-01: la aplicación permitirá registrarse al docente para acceder a la información sobre los cursos.
	\item R-02: la aplicación permitirá al profesor desconectarse.
	\item R-03: la aplicación mostrará un listado de los cursos en los que se encuentre el profesor.
	\item R-04: se deben poder generar informes que evalúen la calidad de los cursos en los que se encuentra el profesor por medio de consultas a diversos aspectos del curso como la corrección a tiempo de las tareas o la correctitud de las fechas.
	\item R-05: se deben poder generar informes globales que resuman los informes de todos los cursos en los que se encuentre el profesor.
\end{itemize}

\section{Especificación de requisitos}
A partir de los requisitos se obtiene el siguiente diagrama de casos de uso:
\imagen{DiagramaCasosUso.PNG}{Diagrama de casos de uso}
 \begin{table}[H]
 	\caption{Caso de uso CU-01}
 	\resizebox{\textwidth}{!}{%
 		\begin{tabular}{cl}
 			\hline
 			\textbf{CU-01}       & \textbf{Registrarse}                                   \\ \hline
 			\textbf{Descripción}          & Permite al usuario registrarse con sus credenciales    \\
 			\textbf{Requisitos asociados} & R-01                                                   \\
 			\textbf{Precondiciones}       & El usuario posee una cuenta valida                     \\
 			\textbf{Ejecución}            &\begin{tabular}[l]{@{}l@{}}1.El usuario introduce su nombre de usuario\\ 2.El usuario introduce su contraseña\\ 3.El usuario modifica el dominio al del\\ servidor al que pretende acceder\\ 4.El usuario selecciona en el menú desplegable\\ el tipo de curso que pretende analizar\\ 5.El usuario hace clic en ``Entrar''\\ 6.El usuario es llevado a la lista de cursos\end{tabular}                                                \\
 			\textbf{Postcondiciones}      &                                                        \\
 			\textbf{Excepciones}          & Las credenciales no son válidas (se indica al usuario)
 		\end{tabular}%
 	}
 \end{table}

 \begin{table}[H]
 	\caption{Caso de uso CU-02}
	\resizebox{\textwidth}{!}{%
		\begin{tabular}{cl}
			\hline
			\textbf{CU-02}       & \textbf{Desconectarse}                                   \\ \hline
			\textbf{Descripción}          & Permite al usuario cerrar la sesión    \\
			\textbf{Requisitos asociados} & R-02                                                   \\
			\textbf{Precondiciones}       & El usuario está logueado en la aplicación                     \\
			\textbf{Ejecución}            &\begin{tabular}[l]{@{}l@{}}1.El usuario hace clic en ``Desconectar'' \\desde la lista de cursos\\ 
			2.Se invalida la sesión del usuario\\
			3.El usuario es llevado a la página de login\end{tabular}                                                \\
			\textbf{Postcondiciones}      & La sesión se ha cerrado correctamente                                                       \\
			\textbf{Excepciones}          & 
		\end{tabular}%
	}
\end{table}

 \begin{table}[H]
 	\caption{Caso de uso CU-03}
	\resizebox{\textwidth}{!}{%
		\begin{tabular}{cl}
			\hline
			\textbf{CU-03}       & \textbf{Mostrar lista de cursos}                                   \\ \hline
			\textbf{Descripción}          & Muestra al usuario el listado de los cursos en los que está   \\
			\textbf{Requisitos asociados} & R-03                                                   \\
			\textbf{Precondiciones}       & El usuario está logueado en la aplicación                     \\
			\textbf{Ejecución}            &\begin{tabular}[l]{@{}l@{}}1.Se utilizan las credenciales para obtener el token de sesión\\ 2.El token se almacena en la sesión\\ 3.Se obtiene el listado de los cursos\\ del usuario utilizando el token\\ 4.La página muestra el listado\end{tabular}                                                \\
			\textbf{Postcondiciones}      &                                                        \\
			\textbf{Excepciones}          & 
		\end{tabular}%
	}
\end{table}

 \begin{table}[H]
 	\caption{Caso de uso CU-04}
	\resizebox{\textwidth}{!}{%
		\begin{tabular}{cl}
			\hline
			\textbf{CU-04}       & \textbf{Generar informe específico}                                   \\ \hline
			\textbf{Descripción}          & Genera un informe del curso correspondiente    \\
			\textbf{Requisitos asociados} & R-04                                                   \\
			\textbf{Precondiciones}       & El usuario está logueado en la aplicación                     \\
			\textbf{Ejecución}            &\begin{tabular}[l]{@{}l@{}}1.El usuario hace clic en el curso correspondiente\\ 2.Se utiliza el token para obtener\\ información sobre el curso\\ 3.Se calculan los porcentajes de desempeño\\ 4.Se añaden los resultados a un archivo csv de registro\\ 5.Se genera el gráfico de evolución a partir\\ del contenido del csv correspondiente\\6.Se muestran los resultados y el gráfico en la página\end{tabular}                                                \\
			\textbf{Postcondiciones}      &\begin{tabular}[c]{@{}c@{}}Se ha generado o ampliado un archivo csv para ese\\ servidor, profesor y curso\end{tabular}                                     \\
			\textbf{Excepciones}          & 
		\end{tabular}%
	}
\end{table}

 \begin{table}[H]
 	\caption{Caso de uso CU-05}
	\resizebox{\textwidth}{!}{%
		\begin{tabular}{cl}
			\hline
			\textbf{CU-05}       & \textbf{Generar informe global}                                   \\ \hline
			\textbf{Descripción}          & Genera un informe a partir del estado general de los cursos    \\
			\textbf{Requisitos asociados} & R-05                                                   \\
			\textbf{Precondiciones}       & El usuario está logueado en la aplicación                     \\
			\textbf{Ejecución}            &\begin{tabular}[l]{@{}l@{}}1.El usuario hace clic en ``Informe general''\\ 2.Se utiliza el token para obtener\\ información sobre todos los curso\\ 3.Se calculan los porcentajes de desempeño medios\\ 4.Se muestran los resultados en la página\end{tabular}                                                \\
			\textbf{Postcondiciones}      &                                                        \\
			\textbf{Excepciones}          & 
		\end{tabular}%
	}
\end{table}
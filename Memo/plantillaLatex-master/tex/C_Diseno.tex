\apendice{Especificación de diseño}

\section{Introducción}
En este apéndice se detalla en qué manera se pretende abordar el conjunto de objetivos y requisitos especificados en el apéndice \ref{apendice:B}. Se definen los datos que utiliza la aplicación, su arquitectura, y su diseño procedimental.
\section{Diseño de datos}
En la siguiente tabla se recogen las entidades asociadas a cada una de las consultas para determinar parte de las clases necesarias para el proyecto.
\begin{table}[H]
	\resizebox{\textwidth}{!}{%
		\begin{tabular}{|l|l|}
			\hline
			\textbf{Consulta}                                                                                                                                        & \textbf{Entidades}        \\ \hline
			Las opciones de progreso del estudiante están activadas                                                                                         & Curso            \\ \hline
			Se proporcionan contenidos en diferentes formatos                                                                                               & Curso, Recurso            \\ \hline
			El curso tiene grupos                                                                                                                           & Grupo, Curso     \\ \hline
			El curso tiene actividades grupales                                                                                                             & Actividad, Curso \\ \hline
			\begin{tabular}[c]{@{}l@{}}Los estudiantes pueden ver las condiciones\\ necesarias para completar una actividad\end{tabular}                    & Curso            \\ \hline
			Todas las actividades tienen la misma nota máxima en el calificador                                                                             & Actividad, Curso            \\ \hline
			Los recursos están actualizados                                                                                                                 & Curso, Recurso        \\ \hline
			Fechas de apertura y cierre de tareas son correctas                                                                                             & Actividad, Curso \\ \hline
			Se detallan los criterios de evaluación (rúbricas, ejemplos)                                                                                    & Curso            \\ \hline
			El calificador no tiene demasiado anidamiento                                                                                                   & Curso            \\ \hline
			Los alumnos están divididos en grupos                                                                                                           & Alumno, Grupo    \\ \hline
			\begin{tabular}[c]{@{}l@{}}El profesor responde en los foros dentro del límite de\\ 48 horas lectivas desde que se plantea la duda\end{tabular} & Curso, Foro          \\ \hline
			Se ofrece retroalimentación de las tareas                                                                                                       & Tarea, Curso     \\ \hline
			Las tareas están calificadas                                                                                                                    & Tarea, Curso     \\ \hline
			El calificador muestra cómo ponderan las diferentes tareas                                                                                      & Curso            \\ \hline
			La mayoría de alumnos responden a los feedbacks                                                                                                 & Feedback, Curso  \\ \hline
			Se utilizan encuestas de opinión                                                                                                                & Survey, Curso    \\ \hline
		\end{tabular}%
	}
\end{table}
La siguiente imagen muestra las relaciones entre entidades que utiliza la aplicación.
\imagen{ModeloDatos.png}{Modelo de datos base}
\section{Diseño procedimental}
\imagen{BocetoInterfaz.png}{Boceto de la interfaz}
El siguiente diagrama de secuencia muestra el proceso de generación de informe específico. La llamada ``descarga de datos'' es una simplificación de más de 10 llamadas que descargan la información necesaria sobre el curso. La llamada ``comprobaciones'' es también una simplificación de las 17 llamadas correspondientes a las comprobaciones.
\imagen{DiagramaSecuencia.png}{Diagrama de secuencia de informe específico}
\section{Diseño arquitectónico}
\subsection{Patrón fachada}
El patrón de diseño fachada consiste en utilizar una clase a modo de interfaz de acceso a un subsistema complejo. La principal ventaja que aprovecha la aplicación de este patrón de diseño es simplificar una serie de funciones del subsistema con un mismo fin en una sola llamada a la fachada. Aparte de esto, reduce el acoplamiento, permitiendo reutilizar partes del subsistema. Un subsistema puede tener varias fachadas con distintas funcionalidades y en caso de añadir un elemento nuevo al subsistema solo sería necesario modificar la fachada.
